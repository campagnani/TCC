% !TeX root = document.tex
% !TeX encoding = UTF-8 Unicode

\begin{abstract}
    %MODELO DE RESUMO PARA TCC
    %Tendo em vista que [justificativa], pesquisa-se sobre [tema], a fim de [objetivo geral]. Para tanto, é necessário [objetivo específico 1], [objetivo específico 2] e [objetivo específico 3]. Realiza-se, então, uma pesquisa [metodologia científica]. Diante disso, verifica-se que [resultado 1], [resultado 2] e [resultado 3], o que impõe a constatação de que [conclusão]. Palavras-chave: [assunto]. [ponto de vista sobre o assunto]. [palavra de ligação entre o assunto e o ponto de vista].
    
    Tendo em vista que fez-se necessário investigar se era possível dar mais liberdade no desenvolvimento de aplicações para um robô industrial com controladora aberta disponível no laboratório de Robótica da unidade de Divinópolis do CEFET-MG. Para isso, foi feito o presente trabalho envolvendo o desenvolvimento de um arranjo cliente servidor. O programa servidor deve ser executado no computador industrial da controladora aberta do robô. O trabalho também prevê a possibilidade do desenvolvimento de aplicações no programa cliente em qualquer linguagem de programação, qualquer arquitetura e qualquer sistema operacional, para ser executado em um terceiro dispositivo. Dessa forma, o desenvolvimento desse arranjo cliente servidor deverá estender a liberdade que a solução da fabricante proporciona aos programadores de seus robôs industriais. Para tanto, foi necessário desenvolver um protocolo de comunicação TCP/IP para enviar e receber variáveis referentes aos ângulos das juntas do robô, desenvolver uma função que salvasse na memória os ângulos de referência das juntas do robô oriundo do software no terceiro dispositivo, e que enviasse os dados sensoriais armazenados na memória para tal software, uma função usa a solução da fabricante para ler dados sensoriais, salvando-os na memória e enviando referências das juntas do robô armazenadas na memória, além de resolver o problema gerado pelo assincronismo da comunicação entre arranjo e controladora com o arranjo e o software no terceiro dispositivo. Realizou-se, então, uma pesquisa de finalidade aplicada, objetivo exploratório, sob o método hipotético-dedutivo, com abordagem quantitativa, realizada com procedimentos experimentais. Diante disso, verifica-se que foi possível realizar a comunicação a uma taxa estável máxima de \SI{2}{\milli\second}, programar e executar o exemplo de software cliente em qualquer linguagem de programação, arquitetura e sistema operacional. Isso levou à constatação de que o objetivo de estender a liberdade que a solução da fabricante proporciona aos programadores de robôs industriais foi atingido.
    
    %RESUMO DO ARTIGO: O presente artigo descreve a implementação de um software servidor, chamado OpenSever, num sistema robótico industrial para facilitar a implementação de malhas de controle externas. O sistema robótico utilizado é composto por uma controladora aberta, um computador industrial externo e um manipulador robótico industrial. A controladora aberta permite o recebimento de comandos de movimentação vindos de softwares compilados no computador industrial fazendo uso da biblioteca \textit{eORL}. Mas este sistema é limitado à capacidade computacional do computador industrial, a sua arquitetura \textit{x86}, ao seu sistema operacional baseado em \textit{Linux}, à linguagem de programação C/C++, que a biblioteca \textit{eORL} é compatível, e à conexão via cabos. A fim de superar as limitações citadas, foi implementado esse software servidor no computador industrial para possibilitar o envio de comandos de movimentação por um software cliente em um dispositivo externo. Tal dispositivo pode ser de qualquer arquitetura, sistema operacional e programado em qualquer linguagem de preferência, desde que tenha a capacidade de se comunicar com o servidor pelo protocolo TCP/IP através do cabo de rede ou conexão sem fio. Além disso, podendo fazer uso de bibliotecas simples de código aberto do sistema operacional. Depois de implementado, foram feitos testes com algumas arquiteturas, sistemas operacionais e linguagens de programação. Ao final, são apresentados alguns dos resultados coletados.
\end{abstract}

Palavras-chave: C5G Open, Manipulador Robótico Industrial, Arranjo cliente servidor

\cleardoublepage{}

\begin{otherlanguage}{english}
	\begin{abstract}
	
	    Considering that it is necessary to investigate if it is possible to give more freedom in the development of applications for an industrial robot with an open controller, it is researched on the possibility of a client-server arrangement in the industrial computer to enable the development of applications in any programming language. programming, any architecture and any operating system on a third device, in order to develop a software client-server arrangement that extends the freedom that the manufacturer's solution provides to programmers of industrial robots. For that, it is necessary to develop a TCP/IP communication protocol to send and receive variables referring to the robot joint angles, to develop a function that saves in the memory the robot joint reference angles from the software in the third device, and that send the sensory data stored in the memory to such software, a function use the manufacturer's solution to read sensory data saving them in the memory and send the robot joint references stored in the memory and solve the problem generated by the asynchronism of the communication between array and controller with the arrangement and software on the third device. A research with an applied purpose, exploratory objective, is then carried out under the hypothetical-deductive method, with a quantitative approach, carried out with experimental procedures. Therefore, it appears that it was possible to communicate at a maximum stable rate of \SI{2}{\milli\second}, program and run the client software in any programming language, architecture and operating system, which requires the verification of that the goal of extending the freedom that the manufacturer's solution provides to industrial robot programmers has been achieved.
	
	
        %This article describes the implementation of a software server, called OpenSever, in an industrial robotic system to facilitate the implementation of external control loops. The robotic system used is composed of an open controller, an external industrial computer and an industrial robotic manipulator. The open controller allows receiving movement commands from software compiled on the industrial computer, making use of the \textit{eORL} library. But this system is limited to the computational capacity of the industrial computer, its architecture \textit{x86}, its operating system based on \textit{Linux}, the programming language C/C++, which the library \textit{eORL} is compatible, and connection via cables. In order to overcome the aforementioned limitations, this software server was implemented in the industrial computer to enable the sending of movement commands by a software client on an external device. Such device can be of any architecture, operating system and programmed in any preferred language, as long as it has the ability to communicate with the server via the TCP/IP protocol through the network cable or wireless connection, and can do so use of simple open source operating system libraries. Once implemented, tests were carried out with some architectures, operating systems and programming languages. Some of the collected results are presented at the end.
    \end{abstract}
    
    Keywords: C5G Open, Industrial Robotic Manipulator, Client server arrangement
\end{otherlanguage}
